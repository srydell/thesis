\section{Summary}
\label{sec:Summary}

In this thesis, two different ways of determining the Hausdorff dimension of largest clusters at $T_c$ using the Worm algorithm was discussed. The scaling method examines the behaviour of the largest cluster as the system size increases. This gave the closest results to the geometrical Hausdorff dimension for a $2D$ Ising cluster of $D_H = 1.38 \pm 0.02$ compared to the analytical answer $D_H = 1.375$ \cite{Duplantier:GeoHausdorff}. Due to time constraints this method was not applied to the $3D$ XY model.

The second method was to approximate the Hausdorff dimension using the box dimension. This gave close results, $D_H = 1.35193 \pm 5 \cdot 10^{-4}$, to the scaling method but seem to diverge from the correct result for small boxes. This method was then applied to a $3D$ XY lattice and gave the results $D_H = 1.807 \pm 5 \cdot 10^{-6}$.

To simplify the simulations, the Villain approximation was used for the energy (See Section \ref{subsec:villainApprox}). The new critical temperature was calculated by measuring the winding number on a range of temperatures. Since the winding number is proportional to the superfluid density, a sharp decrease can be seen around the critical temperature. By varying the system size, the intersections of each drop can be taken as the critical temperature and was calculated as $T_c = 0.3331 \pm 1 \cdot 10^{-4}$.